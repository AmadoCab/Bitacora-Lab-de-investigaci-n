\course{Base de datos de profesores}
\professor{Juan Baldelomar}

\maketitle

\part{Bitácora}
\begin{entry}{4 de febrero}
\tcbsubtitle{\LBlimportant}
\begin{itemize}
    \item Articulo en \href{https://medium.com/codex/principal-component-analysis-pca-how-it-works-mathematically-d5de4c7138e6}{Medium} sobre \textit{PCA}.
\end{itemize}
\tcblower
\tcbsubtitle{\LBlsummary}
El \textit{PCA}, consiste en reducir la dimensión de los vectores de datos que tenemos en nuestro modelo. Esto se consigue calculando el producto punto de cada vector de dato con los \textit{Eigenvectors} de la matriz de covarianza de los datos estandarizados. Matemática desarrollada en \ref{PCA-math}.
\vspace{0.4em}
\tcbsubtitle{\LBltodo}
...
\end{entry}

\newpage

\part{Notas adicionales}
\section{Matemáticas de \textit{PCA}}\label{PCA-math}

\newpage

\tableofcontents
